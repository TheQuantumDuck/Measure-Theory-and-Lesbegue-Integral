\documentclass[../main.tex]{subfiles}

\begin{document}

\section{$\sigma$-algebras}

\begin{definition}\label{def_measure}
A family of subsets $\mathcal{A}\subseteq\mathcal{P}(X)$ where $\mathcal{P}(X)$ is the power set of $X$ is said to be a \textbf{$\sigma$-algebra} if and only if:
\begin{enumerate}
    \item $X\in\mathcal{A}$
    \item $A\in\mathcal{A}\implies X\setminus A\in\mathcal{A}$
    \item $\{A_i\}_{i\in\mathbb{N}}\subseteq\mathcal{A}\implies\bigcup_{i\in\mathbb{N}}A_i\in\mathcal{A}$
\end{enumerate}
\end{definition}

\begin{example}
The power set $\mathcal{P}(X)$ of $X$ is itself a (trivial) $\sigma$-algebra.
\end{example}

\begin{example}
Let $f:X\to Y$ be an arbitrary function and $\mathcal{A}$ be any $\sigma$-algebra over $Y$. Then:
\[
    f^{-1}(\mathcal{A}):=\{f^{-1}(A):A\in\mathcal{A}\},
\]
is a $\sigma$-algebra.
\end{example}

\begin{proposition}
Let $\{\mathcal{A}_{i\in I}\}$ be an arbitrary family of $\sigma$-algebras.
\[
\bigcap_{i\in I}\mathcal{A}_i,
\]
is also a $\sigma$-algebra.
\end{proposition}

\begin{proof}
All the conditions from the definition \ref{def_measure} (of measure) are met by each of sets $A_i$ and thus also by their intersection.
\end{proof}

\begin{definition}
Let $\{A_i\}_{i\in I}$ be an arbitrary family of subsets of $X$. The $\sigma$-algebra $\sigma(\{A_i\}_{i\in I})$ generated by 
\end{definition}

\section{Measure}

\begin{definition}
Let $\mathcal{A}$ be a $\sigma$-algebra over a set $X$. \textbf{Measure} $\mu$ is a function $\mu:\mathcal{A}\to[0,\infty]$ such that if $\{A_i\}_{i\in\mathbb{N}}\subseteq\mathcal{A}$ is a sequence of pairwise disjoint sets, then:

\begin{enumerate}
\item $\mu(\emptyset) = 0$
\item $\mu(\cup_{i\in\mathbb{N}}A_i) = \sum_{i\in\mathbb{N}}\mu(A_i)$
\end{enumerate}

\end{definition}



\end{document}